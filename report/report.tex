\def\year{2020}\relax
%File: formatting-instruction.tex
\documentclass[letterpaper]{article} % DO NOT CHANGE THIS
\usepackage{aaai20}  % DO NOT CHANGE THIS
\usepackage{times}  % DO NOT CHANGE THIS
\usepackage{helvet} % DO NOT CHANGE THIS
\usepackage{courier}  % DO NOT CHANGE THIS
\usepackage[hyphens]{url}  % DO NOT CHANGE THIS
\usepackage{graphicx} % DO NOT CHANGE THIS
\urlstyle{rm} % DO NOT CHANGE THIS
\def\UrlFont{\rm}  % DO NOT CHANGE THIS
\usepackage{graphicx}  % DO NOT CHANGE THIS
\frenchspacing  % DO NOT CHANGE THIS
\setlength{\pdfpagewidth}{8.5in}  % DO NOT CHANGE THIS
\setlength{\pdfpageheight}{11in}  % DO NOT CHANGE THIS
% plot line graphs 
\usepackage{pgfplots, pgfplotstable}
\pgfplotsset{compat=1.16}
\nocopyright
%PDF Info Is REQUIRED.
% For /Author, add all authors within the parentheses, separated by commas. No accents or commands.
% For /Title, add Title in Mixed Case. No accents or commands. Retain the parentheses.
 \pdfinfo{
/Title (Sokoban: PDDL Domain planning with Single-Agent)
/Author (Zhuo Ying Jiang Li,Shumeng Liu,Jiaqing Nie,Wenyu Zheng,Xin Fan Guo)
} %Leave this	
% /Title ()
% Put your actual complete title (no codes, scripts, shortcuts, or LaTeX commands) within the parentheses in mixed case
% Leave the space between \Title and the beginning parenthesis alone
% /Author ()
% Put your actual complete list of authors (no codes, scripts, shortcuts, or LaTeX commands) within the parentheses in mixed case. 
% Each author should be only by a comma. If the name contains accents, remove them. If there are any LaTeX commands, 
% remove them. 

% DISALLOWED PACKAGES
% \usepackage{authblk} -- This package is specifically forbidden
% \usepackage{balance} -- This package is specifically forbidden
% \usepackage{caption} -- This package is specifically forbidden
% \usepackage{color (if used in text)
% \usepackage{CJK} -- This package is specifically forbidden
% \usepackage{float} -- This package is specifically forbidden
% \usepackage{flushend} -- This package is specifically forbidden
% \usepackage{fontenc} -- This package is specifically forbidden
% \usepackage{fullpage} -- This package is specifically forbidden
% \usepackage{geometry} -- This package is specifically forbidden
% \usepackage{grffile} -- This package is specifically forbidden
% \usepackage{hyperref} -- This package is specifically forbidden
% \usepackage{navigator} -- This package is specifically forbidden
% (or any other package that embeds links such as navigator or hyperref)
% \indentfirst} -- This package is specifically forbidden
% \layout} -- This package is specifically forbidden
% \multicol} -- This package is specifically forbidden
% \nameref} -- This package is specifically forbidden
% \natbib} -- This package is specifically forbidden -- use the following workaround:
% \usepackage{savetrees} -- This package is specifically forbidden
% \usepackage{setspace} -- This package is specifically forbidden
% \usepackage{stfloats} -- This package is specifically forbidden
% \usepackage{tabu} -- This package is specifically forbidden
% \usepackage{titlesec} -- This package is specifically forbidden
% \usepackage{tocbibind} -- This package is specifically forbidden
% \usepackage{ulem} -- This package is specifically forbidden
% \usepackage{wrapfig} -- This package is specifically forbidden
% DISALLOWED COMMANDS
% \nocopyright -- Your paper will not be published if you use this command
% \addtolength -- This command may not be used
% \balance -- This command may not be used
% \baselinestretch -- Your paper will not be published if you use this command
% \clearpage -- No page breaks of any kind may be used for the final version of your paper
% \columnsep -- This command may not be used
% \newpage -- No page breaks of any kind may be used for the final version of your paper
% \pagebreak -- No page breaks of any kind may be used for the final version of your paperr
% \pagestyle -- This command may not be used
% \tiny -- This is not an acceptable font size.
% \vspace{- -- No negative value may be used in proximity of a caption, figure, table, section, subsection, subsubsection, or reference
% \vskip{- -- No negative value may be used to alter spacing above or below a caption, figure, table, section, subsection, subsubsection, or reference

\setcounter{secnumdepth}{0} %May be changed to 1 or 2 if section numbers are desired.

% The file aaai20.sty is the style file for AAAI Press 
% proceedings, working notes, and technical reports.
%
\setlength\titlebox{2.5in} % If your paper contains an overfull \vbox too high warning at the beginning of the document, use this
% command to correct it. You may not alter the value below 2.5 in
\title{Sokoban: PDDL Domain planning with Single-Agent}
%Your title must be in mixed case, not sentence case. 
% That means all verbs (including short verbs like be, is, using,and go), 
% nouns, adverbs, adjectives should be capitalized, including both words in hyphenated terms, while
% articles, conjunctions, and prepositions are lower case unless they
% directly follow a colon or long dash
\author{Written by Bachelor students from King's College London \\  % All authors must be in the same font size and format. Use \Large and \textbf to achieve this result when breaking a line
Zhuo Ying Jiang Li - zhuo.jiang\_li@kcl.ac.uk,\textsuperscript{\rm 1} 
Shumeng Liu - shumeng.liu@kcl.ac.uk\textsuperscript{\rm 2} \\ 
Jiaqing Nie - jiaqing.nie@kcl.ac.uk, \textsuperscript{\rm 3} 
Wenyu Zheng - wenyu.zheng@kcl.ac.uk \textsuperscript{\rm 4} \\ 
Xin Fan Guo - xinfan.guo@kcl.ac.uk \textsuperscript{\rm 5} \\ %If you have multiple authors and multiple affiliations
% use superscripts in text and roman font to identify them. For example, Sunil Issar,\textsuperscript{\rm 2} J. Scott Penberthy\textsuperscript{\rm 3} George Ferguson,\textsuperscript{\rm 4} Hans Guesgen\textsuperscript{\rm 5}. Note that the comma should be placed BEFORE the superscript for optimum readability
% email address must be in roman text type, not monospace or sans serif
}
\usepackage{xcolor}
 \begin{document}

\maketitle

\begin{abstract}
Planning in artificial intelligence has been instrumental in a variety of applications. Game serves as a good testing ground for various model planning in PDDL. This paper focuses on PDDL  domain planning and problem analysis of a puzzle-based video game Sokoban.  The reason for this choice is simple, Sokoban is a puzzle-based game that offers intricate dynamics and great complexity for testing planners. Secondly, Sokoban also has some flexibility and offers room for expansion and creativity. As we will see in the following section, we will add our own proposed ideas i.e., tools based on traditional models.  Sokoban problem file ranges with different levels of complicity, thus offers complexity which can be used as the benchmark for evaluating overall plan quality (Sokoban is  a PSPACE-complete problem). The proposed domain will focus on a deterministic model with single-agent and known initial states where state changes are based on the actions proposed by the plan. 
\end{abstract}

\section{Introduction}
This model simulates a well-known grid-based puzzle video game \emph{"Sokoban"}. The original game comes with an objective for a player to push all boxes into predefined storage locations or units. Based on this construct we have added some creativity and enabled tools to be used. The domain represents the 2D \(N*N\) grid with helper tools such as \emph{"trampoline"} which enables the player to jump across the wall or \emph{"bomb"} which enables the player to break walls. Helper tools are placed on grids to be picked up by the user. Because the domain world is represented by a grid with composite squares as single units. The convention used in the problem files are grids being vertically marked with ascending numbers and horizontally with alphabetical characters growing with lexical order (i.e., first most upper left square is represented as \emph{"sq-a1"}, please see Figure \ref{fig:example}). For sake of scalability and variations of problem files, every player has been supplied with number of pre-set \emph{"pliers"} which enables the player to pull a box in any direction. Any of such tools can be used only once and decreases with the use. Except for the trampoline, a player can pick up any number of tools bigger than one. The interesting aspect of this model is that instead of having a concrete goal that defines a box to be at specific positions (i.e., box at square a3). We let planner freely choose to which storage location (represented by a hole in our domain world) player will push the box as long as we allocate all boxes to storage locations(marked as a collected box if the box is pushed on the hole - storage location). Besides, players can re-allocate a box even if it is already on the storage location. Those features add to the flexibility and increment the possibility of finding a plan. For creativity considerations, we have randomly added coins on squares to be picked up. The overall aim of the modified version \emph{"Sokoban 2.0"} is to push all the boxes on storage units with the minimum total cost and the maximum number of coins collected on the way.  
\begin{figure}
    \includegraphics[width=2.8in]{figure1.JPG} \\
    \caption{Example of proposed Sokoban puzzle solved with 33 steps using OPTIC planner}\label{fig:example}
\end{figure}

\section{Part I: Designing a Planning Domain}
In the following section we will pick the most interesting parts of our domain and introduce reasoning behind the design choices and declarations in our domain.pddl file. 

\subsubsection{Domain constants}
To reinforce the idea of a single-agent problem we have declared a player \emph{"p1"} as constant to avoid ambiguity of multi-players. Throughout our problem and domain file this player will be the only actor in our proposed world. 

\subsubsection{Domain predicates}
Probably the most important predicate from our domain is \texttt{(collected-box ?b -box)}. As mentioned, the proposed domain file does restrict itself on specific goals (placing boxes on predefined squares). Instead our aim is to have all declared boxes \emph{"collected"} without explicitly mentioning where to position different boxes. We have in essence given a \emph{free-hand} and left the decision on the planner to find the \emph{"best-optimal solution"}. 

\subsubsection{Domain requirements}
Requirement declaration includes\texttt{ :constraints,:universal-preconditions, :typing,:fluents,:preferences }and\texttt{ :action-costs}. The reason behind declaration of above stated requirements is given as follows:

\begin{itemize}
\item \texttt{:typing} - reflect different object types in the domain i.e., player, box, square as single object declaration can’t represent all different entities in the domain.
\item \texttt{:universal-preconditions} - The use of universal quantifier spared us writing lengthy and tedious preconditions or goals. For instance, instead of listing all collected boxes (boxes on storage units) in conjunction in the form of \texttt{(:goal (and (collected-box \(b_1 ..b_n, b_{n+1}\))))} where \(n >= 1\) we could simply write \texttt{(:goal (forall (?b - box) (collected box)))} indicating that our \emph{hard goal} is to place all existing boxes on some storage units.
\item \texttt{:fluents} - We also exploit the feature of using complete numeric subset of PDDL and use numeric modelling throughout our domain file. Declaring numeric state variables through \texttt{:fluents} allowed numeric formulation of our actions i.e., break-noth-wall with precondition
\texttt{(>(bomb-available ?p) 0)} which restrict the occurrence of the action unless we have more than \(x >= 1\) bombs. Since Sokoban is PSPACE-complete problem, thus problem file evaluation can increase exponentially with size of the search space. For problem file simplification reasons, we have introduced pull action as oppose to traditional Sokoban game which only allows push . Expressed as pull-\emph{direction} with condition \texttt{(>(pliers-available) 0)} indicating that player is able to pull if it has the tool \emph{"plier"}. The number of pre-set pliers initialised in the ground function increases with difficulty of the problem file.  
\item \texttt{:preferences} - Declaration of preferences in the domain file enables us to achieve soft goals and define properties that we would like to achieve but are not compulsory. Thus, we have exploited this feature and defined that it will be preferable for the player to collect as much coins as possible. Implemented through \texttt{(:goal (preference (> (collected-coins) 0))}.
\item \texttt{:action-cost} -  For optimisation reasons, every action in our action schema comes with  a cost which means that every move will incur penalty.  Our goal is to minimize incurred action cost in planning metric so minimum number of steps are taken to reach the objective.  This was realized through declaration of \texttt{:function (total-cost)},\texttt{:effect (increase (total-cost) 1)}and\texttt{ :metric (minimize (total-cost))}. 
\end{itemize}

\subsubsection{Domain actions}
The introduced action schema tries to mimic all possible moves taken by player. Every single action in the domain file is named in self-explanatory way i.e., pull-\emph{direction}, push-\emph{direction}, jump-\emph{direction}, break-\emph{direction} and pick-up-\emph{tool}(bomb, coin, trampoline). Implementing domain file using this kind of architecture allows better trackability of generated plan. It is also important to mention here that push and pull actions come in two different versions i.e., push-\emph{direction}, push-\emph{direction}-to-hole. Former indicating a push to free location , latter indicating a push to hole or storage location (See following section for reason).

\subsubsection{Planner constraints and proposed solutions}
\textbf{OPTIC} planner provided lacks several important functionalities such as support for the \textbf{ADL}, conditional effects and negative preconditions. For such reasons we remodelled the action schema to avoid those features. As some action’s preconditions are made up of a single or clauses of \emph{k} terms \cite{haslum_2019}, we have remodelled the problem to create \emph{k} copies of an action and used one term for each as a regular precondition i.e., instead of writing in the \texttt{:effect (when(hole-at ?box-after-move))} to represent condition when a box is moved on a hole after push or pull. We have re-written this into two push and pull actions i.e., push-up with \texttt{:precondition (no-hole-at ?box-after-move)} and push-up-to-hole with \texttt{:precondition (hole-at ?box-after-move)} and \texttt{:effect (collected-box ?box)}. As mentioned \textbf{OPTIC}  also does not support negative precondition, therefore all predicates which need to be negative in the precondition have been represented by two static predicates i.e., has- trampoline and has-no-trampoline to indicate that a player has this tool. While in principle this does allow \textbf{OPTIC} to work with this problem, but as a result it reduces efficiency and come at expense given increased number of actions required.   

\subsubsection{Possible domain improvements}
Although axioms do not increase language expressiveness, but it will be great to include \emph{derived predicates} as it decreases the domain description size and plan length, compiling away from axiom can result in an exponential growth in number of actions. In our case instead of listing all squares which are above one another in the problem file i.e., \texttt{(above sq-a1 sq-a2)} indicating square \emph{a1} being above \emph{a2}. We could have written \texttt{((: derived (above? a ?b - squares) (and (left-to b?  ?c) (above c? ?a))} that recursively defines the relationship between pairs of squares.   

\section{Part II: Problem analysis}
In this section we will analyse the planner's ability to solve problems in this domain and whether some domain features will
make it harder to solve for the planner. We will also compare the plan quality of the plans to a selected benchmark.

\subsection{Difficulty measure as problem selection criteria}
To select a robust and more complete range of problems in the puzzle domain, we chose to classify the problems according
 to a difficulty measure, and then select a problem suite which covers low difficulty to high difficulty problems. 
 However, finding which problem parameter is the most representative difficulty measure is not an easy task. There are many plausible measures, such as
  the number of steps taken to push all the boxes to the holes, the time taken by a solver (human or artificial 
  intelligence planner) to solve the puzzle or the percentage of successful plays. We will use the first one as our 
  difficulty measure heuristics for this research project for the sake of simplicity.

\subsection{Selected problem suite}
Specifically, we will order the problems according to the
minimum number of steps
taken by a human player to solve it, starting with a 
problem with best score 25 steps up to a problem with 
best score 429 steps. The data of the best scores of
human players are collected from the Sokoban Online website \cite{microban}. 
This is more representative than trying to solve the 
problems ourselves and using our best score to compare 
to that of the OPTIC planner. Given that the Sokoban Online 
community is an active community with global Sokoban 
grandmasters, it is likely that the best score data 
from the website is the global minimum number of steps
to solve each problem. 
   
These problems are, however, just testing a part of
the expressiveness of our domain, because our PDDL
domain can also encode the use of tools --- 
bombs, pliers and trampolines. Thus, we run the same set 
of problems but with added tools (each initially 
with 10 available pliers) to observe how planner 
would handle those additional actions.

Problem 1 to 15 are taken from the forementioned Sokoban Online website, while problem 16 
to 30 are essentially the same 15 problems but provided with 10 pliers. Hence, in following
analysis, problem 16 to 30 are referred to as the "with tool" version of problem 1 to 15.
Problem 31 is an experimental small problem to test trampoline and bomb functionalities
defined in the domain file.

\subsection{Anomalies in difficulty rating}
Before diving into the analysis of the planner's plans, note 
that there may be anomalies in our difficulty assessment 
and the planner's behaviour. There may be some problems 
which are easier to solve according to our difficulty measure but 
the planner is not able to find a solution. While we expect
that the plan quality compared to a benchmark decreases as the
difficulty increases, it is possible to find counterexamples, 
because the difficulty measure (smallest
number of steps required to solve by human players in Sokoban Online) 
is merely a heuristics. 

\subsection{Plan quality measure}
In a Sokoban game, players are rated according to the number
of steps they use to solve the game, and the lowest number 
is marked as their best score. Therefore, using the number of steps 
as plan quality measure fits well our purpose. 

In some online games, players are also ranked according to 
the time taken to solve the level. However, the major problem
with this quality measure is its strong dependence 
on hardware in which the planner is running on. 
It is thus not worth comparing the planner's resolution time 
with human players' average time. 

Another possible quality measure is the number of coins the 
planner collects during its plan trajectory. However, we tested 
the OPTIC planner with a trivial game instance where there is a 
coin in the initial position of the player and the player only 
has to push the box forward to solve the game. It turns out that
the planner's plan does not collect the immediately available coin. 

From what is stated above, we will choose the number of steps 
as the plan quality measure for this reasearch. 
The benchmark is the best player score in Sokoban Online website \cite{microban}.
\subsection{Plan quality analysis}
The diagram in \textbf{Figure 2} illustrates how the solution given by the 
planner (without tools) differs from the benchmark in terms of number 
of steps taken. The trend from problem 1 to problem 14 
suggests that number of steps is effectively a good estimation of 
problem difficulty. The planner can sometimes find the 
optimal solution for problems with under 33 moves. 
Between 40 to 200 steps, even if the problem is solvable 
by the planner, the difference between the benchmark and the 
plan's number of steps is increasing. 
Beyond that, the problem size becomes 
unsolvable. However, the planner managed to solve the
problem 15, when it requires 429 steps for a human player 
to solve it. 
Problem 15 is distinctive because 
it involves only one box, while other problems with difficulty 
heuristics equal to
200-plus steps consist of at least 3 boxes. 

This discovery implies that, the number of steps required is not 
the mere factor that makes a problem difficult, the number of boxes 
or other underlying characteristics of the problem also play a role. 
A similar point is also suggested in a research study: 
there is no single difficulty measure which gives consistent 
evaluation of the actual difficulty for all problem instances
\cite{difficulty-rating}.
% Figure 2
\pgfplotstableread[col sep=&, header=true]{
description&planner&benchmark
1&28&24
2&33&25
3&33&33
4&81&69
5&126&100
6&129&117
7&0&119
8&201&169
9&225&195
10&263&201
11&0&212
12&0&233
13&0&282
14&0&335
15&429&429
}\datatableentry
\begin{figure}
\begin{tikzpicture}
    \begin{axis}[
        enlarge y limits ={value=0.2,upper},
        xtick=data,
        xticklabels ={1,2,3,4,5,6,7,8,9,10,11,12,13,14,15},  
        x tick label style={rotate=-45,anchor=west,font=\tiny},
        legend style={font=\tiny,legend pos=north west,legend cell align=left},
        xlabel = {Problem number},
	      ylabel = {Number of steps},
      ]
      \addlegendentry{Planner};
      \addplot [color=red] table [y=planner, x expr=\coordindex] {\datatableentry};
      \addlegendentry{Benchmark};
      \addplot [color=gray] table [y=benchmark, x expr=\coordindex] {\datatableentry};
    \end{axis}
\end{tikzpicture}
\caption{By number of steps}
\end{figure}

Moreover, for a human player, providing additional tools, such as pliers, would 
only make the problems easier to solve. However, when it comes to planners, 
it seems not the case. When experimenting with the problem files, we noticed 
that, despite it would solve the problem without any tools, the planner is 
unable to give any solution for problems 9 and 10 when provided with 10 
pliers. A plausible explanation is illustrated in in \textbf{Figure 3} and 
\textbf{Figure 4}.
% Figure 3
\pgfplotstableread[col sep=&, header=true]{
description&with-tools&without-tools
1&147&106
2&87&1150
3&35&396
4&12889&6475
5&137535&7276
6&87887&89873
7&21992&0
8&97430&55628
9&0&41715
10&0&31984
11&0&0
12&0&0
13&0&0
14&0&0
15&50172&6150
}\datatableentry
\begin{figure}
\begin{tikzpicture}
    \begin{axis}[
        enlarge y limits ={value=0.2,upper},
        xtick=data,
        xticklabels ={1,2,3,4,5,6,7,8,9,10,11,12,13,14,15},  
        x tick label style={rotate=-45,anchor=west,font=\tiny},
        legend style={font=\tiny,legend pos=north west,legend cell align=left},
        xlabel = {Problem number},
	      ylabel = {Solve time},
      ]
      \addlegendentry{Planner with tools};
      \addplot [color=blue] table [y=with-tools, x expr=\coordindex] {\datatableentry};
      \addlegendentry{Planner without tools};
      \addplot [color=red] table [y=without-tools, x expr=\coordindex] {\datatableentry};
    \end{axis}
\end{tikzpicture}
\caption{By solve time}
\end{figure}
% ADD a column of total states in search space 
% search space = 2^{number predicates}
% number predicates = |square|+|box|*|square|+|square|+|bomb|+|coin|+|square|*|square|*2+|square|+|square|+|box|+|square|+1
%                   = (|box|+2*|square|+5)*|square| + |bomb| + |coin| + |box| + 1
% Problem 1: |square| = 36, |box| = 2, search space = overflow
% Figure 4
\pgfplotstableread[col sep=&, header=true]{
description&with-tools&without-tools
1&0.13&0.11
2&0.1&1.73
3&0.03&0.28
4&44.31&21.84
5&533.50&25.53
6&301.88&358.41
7&88.38&0
8&397.06&297.89
9&0&274.08
10&0&205.84
11&0&0
12&0&0
13&0&0
14&0&0
15&479.55&48.96
}\datatableentry
\begin{figure}
\begin{tikzpicture}
    \begin{axis}[
        enlarge y limits ={value=0.2,upper},
        xtick=data,
        xticklabels ={1,2,3,4,5,6,7,8,9,10,11,12,13,14,15},  
        x tick label style={rotate=-45,anchor=west,font=\tiny},
        legend style={font=\tiny,legend pos=north west,legend cell align=left},
        xlabel = {Problem number},
	      ylabel = {Number of states evaluated},
      ]
      \addlegendentry{Planner with tools};
      \addplot [color=blue] table [y=with-tools, x expr=\coordindex] {\datatableentry};
      \addlegendentry{Planner without tools};
      \addplot [color=red] table [y=without-tools, x expr=\coordindex] {\datatableentry};
    \end{axis}
\end{tikzpicture}
\caption{By number of states evaluated}
\end{figure}
These two diagrams show that, due to more actions become possible, 
when provided with additional tools, planner would evaluate more 
states and take a longer time to derive a solution. This implies 
the planner fails to effectively prune out bad uses of the tool and 
wasted a lot of resources exploring unsolvable states. 
\subsection{Plan quality analysis improvements}
The only measure of the plan quality was whether 
the designed plan solves the problem and how many 
steps have been taken to achieve the goal. Perhaps a better 
evolution strategy is to define a set of preferable 
soft goals which is preferable to achieve but not 
compulsory. As the plan results in a higher number 
of correctly allocated boxes, the higher the quality 
of the plan.  This is helpful to analyse problems 
of higher difficulty in more detail rather than 
treating them all as unsolvable. We can observe 
how the plan quality over the benchmark increases 
or deteriorates, based on the number of achieved 
soft goals.  

If OPTIC supports \texttt{:preference} and \texttt{forall}
PDDL features we could write 
\begin{verbatim}
(preference allBoxesAllocated 
  (forall (?b - box) (collected-box ?b))
)  
\end{verbatim}
or 
\begin{verbatim}
(forall (?b - box) 
  preference allocated (collected-box ?b)
)
\end{verbatim}
, the former incurring a penalty if any box is
 not allocated and the latter incurring a 
 penalty for each non-allocated box. 
 The overall quality measure is given by the metric
\begin{verbatim}
(:metric minimize 
  (* 2 (is-violated allocated)
  (*100 (is-violated allBoxesAllocated)
)
\end{verbatim}
 giving a linear combination of the preference of plan. 
 A plan that reaches a goal state satisfying 
 all preferences will have penalty of 0.

\bibliographystyle{aaai}
\bibliography{references}

\end{document}
